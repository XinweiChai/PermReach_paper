% Styles TikZ et couleurs personnalisées

\usepackage{tikz}

\newdimen\pgfex
\newdimen\pgfem
\usetikzlibrary{arrows,shapes,shadows,scopes}
\usetikzlibrary{positioning}
\usetikzlibrary{matrix}
\usetikzlibrary{decorations.text}
\usetikzlibrary{decorations.pathmorphing}
\usetikzlibrary{arrows,shapes}

\definecolor{lightgray}{rgb}{0.8,0.8,0.8}
\definecolor{lightgrey}{rgb}{0.8,0.8,0.8}

\definecolor{lightred}{rgb}{1,0.8,0.8}
\definecolor{lightgreen}{rgb}{0.7,1,0.7}
\definecolor{darkgreen}{rgb}{0,0.5,0}
\definecolor{darkblue}{rgb}{0,0,0.5}
\definecolor{darkyellow}{rgb}{0.5,0.5,0}
\definecolor{lightyellow}{rgb}{1,1,0.6}
\definecolor{darkcyan}{rgb}{0,0.6,0.6}
\definecolor{lightcyan}{rgb}{0.6,1,1}
\definecolor{darkorange}{rgb}{0.8,0.2,0}
\definecolor{notsodarkred}{rgb}{0.8,0,0}

\definecolor{notsodarkgreen}{rgb}{0,0.7,0}

%\definecolor{coloract}{rgb}{0,1,0}
%\definecolor{colorinh}{rgb}{1,0,0}
\colorlet{coloract}{darkgreen}
\colorlet{colorinh}{red}
\colorlet{coloractgray}{lightgreen}
\colorlet{colorinhgray}{lightred}
\colorlet{colorinf}{darkgray}
\colorlet{coloractgray}{lightgreen}
\colorlet{colorinhgray}{lightred}

\colorlet{colorgray}{lightgray}
\colorlet{colorhl}{blue}


\tikzstyle{boxed ph}=[]
\tikzstyle{sort}=[fill=lightgray, rounded corners, draw=black]
\tikzstyle{process}=[circle,draw,minimum size=15pt,fill=white,font=\footnotesize,inner sep=1pt]
%\tikzstyle{black process}=[process, draw=blue, fill=red,text=black,font=\bfseries]
\tikzstyle{gray process}=[process, draw=black, fill=lightgray]
\tikzstyle{highlighted process}=[current process, fill=gray]
\tikzstyle{process box}=[fill=none,draw=black,rounded corners]
\tikzstyle{current process}=[process, draw=black, fill=lightgray]
%\tikzstyle{current process}=[process,fill=lightcyan]
\tikzstyle{hl process}=[process,fill=blue!30]
\tikzstyle{tick label}=[font=\footnotesize]
\tikzstyle{tick}=[densely dotted] %-
\tikzstyle{hit}=[->,>=angle 45]
\tikzstyle{selfhit}=[min distance=50pt,curve to]
\tikzstyle{bounce}=[densely dotted,>=stealth',->]
\tikzstyle{ulhit}=[draw=lightgray,fill=lightgray]
\tikzstyle{pulhit}=[fill=lightgray]
\tikzstyle{bulhit}=[draw=lightgray]
\tikzstyle{hl}=[very thick,colorhl]
\tikzstyle{hlb}=[very thick]
\tikzstyle{hlhit}=[hl]
%\tikzstyle{hl2}=[hl]
%\tikzstyle{nohl}=[font=\normalfont,thin]

\tikzstyle{update}=[draw,->,dashed,shorten >=.7cm,shorten <=.7cm]

\tikzstyle{unprio}=[draw,thin]%[double]
%\tikzstyle{prio}=[draw,thick,-stealth]%[double]
\tikzstyle{prio}=[draw,-stealth,double]

\tikzstyle{hitless graph}=[every edge/.style={draw=red,-}]

\tikzstyle{aS}=[every edge/.style={draw,->,>=stealth}]
\tikzstyle{Asol}=[draw,circle,minimum size=5pt,inner sep=0,node distance=1cm]
\tikzstyle{Aproc}=[draw,node distance=1.2cm]
\tikzstyle{Aobj}=[node distance=1.5cm]
\tikzstyle{Anos}=[font=\Large]

\tikzstyle{AsolPrio}=[Asol,double]
\tikzstyle{AprocPrio}=[Aproc,double]
\tikzstyle{aSPrio}=[aS,double]

\colorlet{colorhlwarn}{notsodarkred}
\colorlet{colorhlwarnbg}{lightred}
\tikzstyle{Ahl}=[very thick,fill=colorhlwarnbg,draw=colorhlwarn,text=colorhlwarn]
\tikzstyle{Ahledge}=[very thick,double=colorhlwarnbg,draw=colorhlwarn,color=colorhlwarn]





%\definecolor{darkred}{rgb}{0.5,0,0}



\tikzstyle{grn}=[every node/.style={circle,draw=black,outer sep=2pt,minimum
                size=15pt,text=black}, node distance=1.5cm, ->]
\tikzstyle{inh}=[>=|,-|,draw=colorinh,thick, text=black,label]
\tikzstyle{act}=[->,>=triangle 60,draw=coloract,thick,color=coloract]
\tikzstyle{inhgray}=[>=|,-|,draw=colorinhgray,thick, text=black,label]
\tikzstyle{actgray}=[->,>=triangle 60,draw=coloractgray,thick,color=coloractgray]
\tikzstyle{inf}=[->,draw=colorinf,thick,color=colorinf]
%\tikzstyle{elabel}=[fill=none, above=-1pt, sloped,text=black, minimum size=10pt, outer sep=0, font=\scriptsize,draw=none]
\tikzstyle{elabel}=[fill=none,text=black, above=-2pt,%sloped,
minimum size=10pt, outer sep=0, font=\scriptsize, draw=none]
%\tikzstyle{elabel}=[]


\tikzstyle{plot}=[every path/.style={-}]
\tikzstyle{axe}=[black,->,>=stealth']
\tikzstyle{ticks}=[font=\scriptsize,every node/.style={black}]
\tikzstyle{mean}=[thick]
\tikzstyle{interval}=[line width=5pt,red,draw opacity=0.7]
%\definecolor{lightred}{rgb}{1,0.3,0.3}

%\tikzstyle{hl}=[yellow]
%\tikzstyle{hl2}=[orange]

%\tikzstyle{every matrix}=[ampersand replacement=\&]
%\tikzstyle{shorthandoff}=[]
%\tikzstyle{shorthandon}=[]
\tikzstyle{objective}=[process,very thick,fill=yellow!50]

\tikzstyle{coopupdate}=[-stealth,decorate,decoration={zigzag,amplitude=1.5pt,post=lineto,post length=.3cm,pre=lineto,pre length=.3cm}]

\tikzstyle{labelprio}=[circle, fill=blue!30, inner sep=0pt, minimum size=13pt]
\tikzstyle{labelprio1}=[labelprio]
\tikzstyle{labelprio2}=[labelprio, fill=red!60]
\tikzstyle{labelprio3}=[labelprio, fill=orange!50]
\tikzstyle{labelprio4}=[labelprio, fill=brown!50]

\tikzstyle{labelstocha}=[rectangle, rounded corners=4pt]

\tikzstyle{andot}=[circle, fill=black, inner sep=1.2pt, draw=transparent]
\tikzstyle{anligne}=[thick]

\tikzstyle{apdot}=[andot] %[circle, fill=black, draw=black, inner sep=1]
\tikzstyle{apdotsimple}=[] %[circle, fill=black, draw=black, inner sep=1]

% Figure de résumé des liens entre les formalismes
\tikzstyle{equiv-externe}=[thick, rounded corners, draw=gray, fill=gray!10, align=center,
  inner sep=8]

% label pour les délais des actions 
 \tikzstyle{labeldelai1}=[circle, fill=red!60, inner sep=0pt, minimum size=8pt]
  \tikzstyle{labeldelai2}=[circle, fill=blue!30, inner sep=0pt, minimum size=8pt]
  \tikzstyle{labeldelai3}=[circle, fill=brown!50, inner sep=0pt, minimum size=8pt]
  \tikzstyle{labeldelai4}=[circle, fill=green!50, inner sep=0pt, minimum size=8pt]
  
% Automata Networks:
\tikzstyle{local transitions}=[->,>=latex',thick,bend left=30,
               every node/.style={fill=white,inner sep=1pt,outer sep=1pt}]
\tikzstyle{reach}=[fill=lightgray,ellipse]

\tikzstyle{local transitions 2}=[->,>=latex',thick,bend left=100,
               every node/.style={fill=white,inner sep=1pt,outer sep=4pt}]

\tikzstyle{local transitions 3}=[->,>=latex',thick,bend right=100,
               every node/.style={ right, fill=white,inner sep=1pt,outer sep=4pt}]
               
% Graphe d'états
% noeuds
\tikzstyle{vide}= [rectangle, minimum width=2em,minimum height=1.5em,]
\tikzstyle{stable}= [rectangle,fill=lightred]
\tikzstyle{current}= [rectangle,fill=lightcyan]
\tikzstyle{initial}= [rectangle,fill=green!20]

% % STG transitions
\tikzstyle{etiquette}=[midway,fill=blue!20,circle,scale=0.7pt]
\tikzstyle{etiquette2}=[midway,fill=green!20,circle,scale=0.7pt]
%
\tikzstyle{currentTrans}=[->,very thick,blue]
\tikzstyle{seperatedTransPart1}=[draw, thick, blue]
\tikzstyle{seperatedTransPart2}=[->, thick, blue]


\tikzstyle{mytext}=[thick, text width=4.5em,inner sep=1pt]
\tikzstyle{line} =[draw, thick, -latex',shorten >=2pt]
\tikzstyle{block} =[rectangle,text width=6em,draw,minimum height=4em, outer sep=0pt]

\tikzstyle{adn}=[every node/.style={circle,draw=black,outer sep=2pt,minimum
                size=15pt,text=black}, node distance=1.5cm, ->]
                
% Définition des nouvelles options xmin, xmax, ymin, ymax
% Valeurs par défaut : -3, 3, -3, 3
\tikzset{
    xmin/.store in=\xmin, xmin/.default=-3, xmin=-3,
    xmax/.store in=\xmax, xmax/.default=3, xmax=3,
    ymin/.store in=\ymin, ymin/.default=-3, ymin=-3,
    ymax/.store in=\ymax, ymax/.default=3, ymax=3,
}
% Commande qui trace la grille entre (xmin,ymin) et (xmax,ymax)
\newcommand {\grille}
    {\draw[help lines] (\xmin,\ymin) grid (\xmax,\ymax);}
% Commande \axes
\newcommand {\axes} {
    \draw[->] (\xmin,0) -- (\xmax+0.5,0);
    \draw[->] (0,\ymin) -- (0,\ymax+0.5);
}
% Commande qui limite l’affichage à (xmin,ymin) et (xmax,ymax)
\newcommand {\fenetre}
    {\clip (\xmin,\ymin) rectangle (\xmax,\ymax);}
   
 \newcommand{\nombresCopiesParNote}
   {(0,0)(1,0)(2,2)(3,0)(4,6)(5,4)(6,7)(7,4)(8,3)(9,0)(10,1)}  
% Expression level of a
 \newcommand{\expressionDiscreteA}
   {(0,0)(1,0)(2,0)(3,0)(4,0)(4,1)(5,1)(5,0)(6,0)(7,0)(8,0)(9,0)(9,1)(10,1)(11,1)(12,1)(13,1)(13,0)(14,0)(15,0)(16,0)(17,0)(17,1)(18,1)}      
   
% Expression level of b
 \newcommand{\expressionDiscreteB}
   {(0,1)(12,1)(12,0)(18,0)} 
   
% Expression level of z
 \newcommand{\expressionDiscreteZ}
   {(0,0)(3,0)(3,1)(14,1)(14,0)(18,0)}